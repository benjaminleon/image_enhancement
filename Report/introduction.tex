\chapter{Introduction}\label{cha:intro}
Back in the days when all photographies were analogue, several hours were usually put into the development of the photo after the camera button was clicked. Images were rare then, and in some sense more precious. 

As cameras became cheaper, more people could afford them and more images were produced. Today, images are in abundance, and spending hours on post-production on images is practised exclusively by professionals photographers or the rare enthusiast. 

Many images are taken by amateurs, with cheap gear which comes included in their mobile phone. We all want to take good pictures, but don't feel like learning the details about how to make the best use your our cameras. The 

But the need for quickly enhancing the quality of an image is not reserved to the for the amateur; Even professionals find a need for a swift edit of their digital photos. 

Instagram, snapchat, and similar services put a lot of effort into developing post-processing techniques for helping their users to make the most out of their digital photos. Within such an application, a single touch on the screen is often enough to give the image a substantial improvement. 

In this thesis, automatic image improvement will be explored with limitations to colorization and super-resolution. Artificial neural networks are an interesting sub-field within machine-learning and artificial intelligence, concepts which are of great importance today, and will probably be much, much more in the future. 

Artificial neural networks recently received an upswing in popularity because they have been shown to be able to learn how to perform well on a range of certain tasks, including image improvement. The ultimate goal of this thesis is to find a neural network which "understands" how we humans like images to look, and given an image it has never seen before, produce a better version of it. One could say that we train an AI to make educated guesses on how enhance image quality.

